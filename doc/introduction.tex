Signals can be observer all around us. 
For example, measuring the time taken between a weight-driven
pendulum clock's ticks produces a signal. It does not require 
a great deal of effor to image how such a signal behaves.
We would expect the clock's pendulum to swing back and forth,
each time travelling a minutely shorter distance until the pendulum
stops completely. Analysis of even a part of such a signal can help 
us determine the pendulum's position far into future.


Now consider a more complex siganl: the rates of a stock market.
People have been analyzing this data for decades, grasping to 
predict its future state.
For the scope of this paper, we defined the term signal processing as
\textit{the science of analyzing time-varying processes} \cite{lyons2004understanding}.


In this thesis we analyzed the non-triviality of digital sygnals.
Certain signals can be classified as simple (relatively trivial),
like the aforementioned clock's pendulum.
A more complex (non-trivial) signal would be the rates of a stock exchange.
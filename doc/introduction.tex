Humans are not particularly good at dealing with large quantities of data, especially when it is expressed in a numeric value.
We are visual creatures, as the common expression states \emph{I Won’t Believe It Until I See It}.
This has lead to an explosion in visualization tools and methods.
Depending on what the illustration is intented to represent, unique techniques are used.
These techniques help us see the data from a unique perspective by emphasising certain aspects of data.
For example in order to display possible logical relations between a colletion of data sets one might choose the venn diagram~\cite{venn-diagram}.

A task at hand is the visualization of data randomness and regularities.
The common method used for exactly this job is a recurrence plot.
This data analysis tool evaluates a stream of data and produces and image.
For an untrained eye the visualization might seem trivial, but after seing a few examples one does not require a great deal of effort to learn how to read the recurrence plot.
The algorithm used to generate this image has its peculiarities but the paper seeks to clarify them.

On the other hand, for a long time visual tasks were performed exclusively by humans.
The underlying principles for machine learning have been around for quite some time, but were limited by hardware.
Rapid increase in affordable computational power coupled with open sourced software is enabling the widespread adoption of this technology.
This is an attempt to apply the current advances in machine learning in order to tackle a task of classifying characteristics of a recurrence plot.
For the model to be utilized, a web application infrastructure is build allowing one to explore the recurrence plot and its properties.
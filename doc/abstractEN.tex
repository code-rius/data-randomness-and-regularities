Šio projekto tikslas buvo sukurti interneto aplikaciją, kuri sugebėti sukurti bei kategorizuoti rekurencines diagramas naudojamas duomenų duomenų dėsningumų ir atsitiktinų vizualizavimui.
Aplikacija sugeba reaguoti į varotojo įrenginio dydį.
Dėl šios priežasties ją galima naudotis tiek mobiliuoju telefonu, tiek namų kompiuteriu.
Duomenu kategorizacija yra įgyvendinta panaudojant kovoliucinius neuroninius tinklus.
Neuroninis tinklas gali tiksliai klasifikuoti chaotinius, periodinius bei trendą turinčius duomenis.
Ši publikacija detaliai aprašo aplikacijos vystymo procesą.
Ji taip pat nupasakoja kaip generuojami duomenys kovoliucinio neuroninio tinklo mokymams bei testavimui.
Galiausiai, buvo prieita išvada, kad konvoliucinis neuroninis tinklas gali išmokti kategorizuoti rekurentines diagramas. 
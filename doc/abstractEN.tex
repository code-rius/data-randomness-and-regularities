Šio projekto tikslas buvo sukurti interneto aplikaciją, kuri sugebėtų atpažinti bei kategorizuoti rekurentines diagramas.
Šios diagramos yra naudojamos atpažinti duomenų  dėsningumus ir atsitiktinumus.
Aplikacija buvo sėkmingai sukurti. Ji sugeba reaguoti į įrenginio dydį.
Dėl šios priežasties ja galima naudotis tiek mobiliuoju telefonu, tiek namų kompiuteriu.
Duomenų kategorizacija įgyvendinta panaudojant konvoliucinius neuroninius tinklus.
Sukurtas tinklas gali tiksliai klasifikuoti chaotinius, periodinius bei trendą turinčius duomenis.
Ši publikacija detaliai aprašo aplikacijos vystymo procesą.
Ji taip pat nupasakoja kaip generuojami duomenys kovoliucinio neuroninio tinklo mokymui bei testavimui.
Galiausiai, buvo prieita išvada, kad konvoliucinis neuroninis tinklas gali išmokti kategorizuoti rekurentines diagramas. 
% Kompiuterijos katedros šablonas
% Template of Department of Computer Science II
% Versija 1.0 2015 m. kovas [ March, 2015]

\documentclass[a4paper,12pt,fleqn]{article}
\input{allPacks}

\newtoggle{inLithuanian}
 %If the report is in Lithuanian, it is set to true; otherwise, change to false
\settoggle{inLithuanian}{false}

%create file preface.tex for the preface text
%if preface is needed set to true
\newtoggle{needPreface}
\settoggle{needPreface}{false}

\newtoggle{signaturesOnTitlePage}
\settoggle{signaturesOnTitlePage}{true}

\input{macros}

\begin{document}
 % #1 -report type, #2 - title, #3-7 students, #8 - supervisor
 \depttitlepage{Bachelors Thesis}{Implementation of application for visualization of regularities and randomness in data}{Audrius Baranauskas} 
 {}{}{}{}% students 2-5
 {dr. Tadas Meškauskas}

\tableofcontents


%keywords and notations if needed
\sectionWithoutNumber{Keywords}{keywords}{Pateikiamas terminų sąrašas (jei reikia)}

 %both abstracts
\bothabstracts{The goal of this project was to create a web application for generating and classifying recurrence plots that are used to visualize data regularities and randomness.
The application was successfully developed and features a responsive design allowing access from desktop and mobile users.
Classification feature is implemented by training a convolutional neural network.
It can effectively distinguish chaotic, period and trending data types.
The paper dives into the application development process.
It also details the convolutional neural network data generation, training and testing procedures.
A conclusion was reached that a convolutional neural network can identify data characteristic by analyzing a recurrence plot image.}%tex-file of abstract in original language
{Darbo pavadinimas kita kalba} %if work is in LT this title should be in English
{Šio projekto tikslas buvo sukurti interneto aplikaciją, kuri sugebėti sukurti bei kategorizuoti rekurencines diagramas naudojamas duomenų duomenų dėsningumų ir atsitiktinų vizualizavimui.
Aplikacija sugeba reaguoti į varotojo įrenginio dydį.
Dėl šios priežasties ją galima naudotis tiek mobiliuoju telefonu, tiek namų kompiuteriu.
Duomenu kategorizacija yra įgyvendinta panaudojant kovoliucinius neuroninius tinklus.
Neuroninis tinklas gali tiksliai klasifikuoti chaotinius, periodinius bei trendą turinčius duomenis.
Ši publikacija detaliai aprašo aplikacijos vystymo procesą.
Ji taip pat nupasakoja kaip generuojami duomenys kovoliucinio neuroninio tinklo mokymams bei testavimui.
Galiausiai, buvo prieita išvada, kad konvoliucinis neuroninis tinklas gali išmokti kategorizuoti rekurentines diagramas. }%tex-file of abstract in other language


 %Introduction section: label is sec:intro
\sectionWithoutNumber{\keyWordIntroduction}{intro}
Signals can be observer all around us. 
For example, measuring the time taken between a weight-driven
pendulum clock's ticks produces a signal. It does not require 
a great deal of effor to image how such a signal behaves.
We would expect the clock's pendulum to swing back and forth,
each time travelling a minutely shorter distance until the pendulum
stops completely. Analysis of even a part of such a signal can help 
us determine the pendulum's position far into future.


Now consider a more complex siganl: the rates of a stock market.
People have been analyzing this data for decades, grasping to 
predict its future state.
For the scope of this paper, we defined the term signal processing as
\textit{the science of analyzing time-varying processes} \cite{lyons2004understanding}.


In this thesis we analyzed the non-triviality of digital sygnals.
Certain signals can be classified as simple (relatively trivial),
like the aforementioned clock's pendulum.
A more complex (non-trivial) signal would be the rates of a stock exchange.



%the main part
\newpage

\section{Recurrence diagram}
\label{sec:motivation}

\subsection{Signals and signal processing}

\begin{lstlisting}
  sygnalStates = []

  # Generate data pairs, tripplets, quadruplets... D - plets
  for i in range(0, self.M):
      state = []

      for j in range(0, self.D):
          state.append(self.data[i+(j*self.d)])

      sygnalStates.append(state)
\end{lstlisting}
% 1. Rekurencine diagrama
% 1.1. Signalu analize
% 1.2. 
%
%


\section{Pirmasis skyrius}
\label{sec:motivation}
\subsection{Pirmojo skyriaus poskyris}
\label{sec:example}
Pateikiamas \ref{sec:example} poskyrio tekstas. Vienas iš šaltinių~\cite{KTZ}. Visas turinys priklauso \ref{sec:motivation} skyriui.

\subsubsection{Pirmojo skyriaus pirmo poskyrio poskyris}
\label{sec:data}
Pateikiamas trečio lygio poskyrio tekstas.

\begin{equation}
x = \sum_{i=1}^N m_i
\end{equation}

\begin{table}[!ht]\centering
\caption{Lentelė ... }
\label{tabl:table}
\begin{tabular}{l|r|}
test&test\\ \hline
test&test\\
\end{tabular}
\end{table}

Sprendimas pristatomas \ref{alg:1} algoritme, o įgyvendinimas -- \ref{abc} išeities kode.

\begin{algorithm}\caption{Algoritmas uždavinio sprendimui}
  \label{alg:1}
  \begin{algorithmic}
    \REQUIRE 
    \ENSURE 
\STATE a \AND b
\end{algorithmic}


\end{algorithm}



\begin{lstlisting}[caption={Pagrindinio metodo žingsniai},label={abc}]
public static void main(String args[]){
}
\end{lstlisting}

 %Conclusions section
\sectionWithoutNumber{\keyWordConclusions}{conclu}
During the research and development phases of this project, the following goals were achieved:
\begin{itemize}
    \item The recurrence plot was analysed to evaluate what can be determinen from the visualization. Several observable characteristics were distinguished. A local implementation of the recurrence plot was developed and utilized throughout the application.
    \item A convolutional neural network, capable of identifying the characteristics of a recurrence plot, was developed and trained. The trained data model performed with high accuracy.
    \item An web application for generating recurrence plots was developed. A microservice architecture was used develop the application providing a modern, scalable solution. The application utilizes the aforementioned CNN to provide information about the generated, that a novel user might not distinguish.
\end{itemize}

The application functionality can be yet improved. Microservice could be containerized to provide easier deployment. The front end part of the application could support full CRUD operations for managing of data in the database. So far this this can only be done by accessing the backend service via an exposed API.

%ateities darbų gairės, planas/next steps of the work
\sectionWithoutNumber{Ateities tyrimų planas}{future}{Pristatomi ateities darbai ir/ar jų planas, gairės tolimesniems darbams....}


 %file literatureSources.bib
\referenceSources{literatureSources}



%% this part is optional
\newpage
\begin{appendices}
Dokumentą sudaro du priedai: \ref{app:a} priede  ....
\newpage
\section{Pirmojo priedo pavadinimas}
\label{app:a}
Pirmojo priedo tekstas ...

\newpage
\section{Antrojo priedo pavadinimas}
Antrojo priedo tekstas ...

\end{appendices}


\end{document}
